\documentclass[usenames,dvipsnames,oneside,10.9pt]{article}

\setlength\evensidemargin{30mm}

\usepackage{graphicx}
\usepackage[utf8]{inputenc}
\usepackage[IL2]{fontenc} %font pro češtinu
\usepackage[czech]{babel}
\usepackage{amsmath, bm} %matika
\usepackage{hyperref} %reference
\usepackage{subcaption}
\usepackage{float} %přichycování obrázků ([H] - HERE)
\usepackage{color} %barevný text
\usepackage{titling}
\usepackage{booktabs}
\usepackage[backend=bibtex8,sorting=ynt]{biblatex}
%\usepackage{DejaVuSansCondensed}
%\renewcommand*\familydefault{\sfdefault} %% Only if the base font of the document is to be sans serif
%\usepackage[T1]{fontenc}
\usepackage[math]{kurier}
\usepackage[T1]{fontenc}

\begin{document}
\thispagestyle{empty}
\vspace*{-2.9cm}
\hspace*{-3cm}
\begin{minipage}[c]{0.25\textwidth}
\begin{figure}[H]
\centering
\includegraphics[width=\textwidth]{./fig/logo_fzp.png}
\end{figure}
\end{minipage}
\hspace*{9mm}
\begin{minipage}[c]{0.75\textwidth}
\bfseries{Česká zemědělská univerzita v Praze} \\

\bfseries{Fakulta životního prostředí}
\end{minipage}

\vspace*{0.15cm}

\centerline{\large \textbf{ZADÁNÍ BAKALÁŘSKÉ PRÁCE}}

\vspace*{0.6cm}

\hspace*{-3cm}
\begin{tabular}{ll}

\noalign{\vspace{2mm}}
Autorka práce: \hspace*{2.3cm} & Irina Georgievová \\
\noalign{\vspace{2.3mm}}
Studijní program: \hspace*{2.3cm} & Krajinářství \\
\noalign{\vspace{2mm}}
Obor: \hspace*{2.3cm} & Vodní hospodářství \\
\end{tabular}

\vspace*{2mm}

\hspace*{-3cm}
\begin{tabular}{ll}

\noalign{\vspace{2mm}}
Vedoucí práce: \hspace*{1.55cm} & doc. Ing. Martin Hanel, Ph.D. \\
\noalign{\vspace{2mm}}
Garantující pracoviště: \hspace*{1.55cm} & Katedra vodního hospodářství a environmentálního modelování \\
\noalign{\vspace{2mm}}
Jazyk práce: \hspace*{1.55cm} & Čeština \\
\end{tabular}

\vspace*{0.35cm}

\hspace*{-3cm}
\begin{tabular}{ll}

\noalign{\vspace{2mm}}
Název práce: \hspace*{1.2cm} & {\large \textbf{Vizualizace environmentálních dat}} \\
\noalign{\vspace{3mm}}
Název anglicky: \hspace*{2.1cm} & \textbf{Visualization of environmental data} \\
\end{tabular}

\vspace*{4mm}

\hspace*{-2.7cm} Cíle práce: \hspace*{3.3cm} \parbox[t]{0.95\textwidth}{Představení klíčových poznatků týkajících se vizualizace a průzkumové analýzy dat z teoretického hlediska i z hlediska praktické implementace v R. Zhodnoceny budou jak nástroje obsažené v základní distribuci R, tak nástroje dostupné v balících lattice, grid, ggplot2, raster, rasterVis, případně i nástroje pro tvorbu dynamických vizualizací (htmlwidgets, shiny apod.).} \\

\vspace*{3mm}

\hspace*{-2.7cm} Metodika: \hspace*{3.3cm} \parbox[t]{0.95\textwidth}{- rešerše základních poznatků \\
- popis vizualizačních prostředků se zaměřením na využití v hydrologii, \\
\hspace*{2mm}porovnání výhod/nevýhod \\
- popis nejpoužívanějších R balíků, jejich základních funkcí a demonstrace\\
\hspace*{2mm}jejich využití} \\

\vspace*{3mm}

\hspace*{-2.7cm} Doporučený rozsah práce: \hspace*{0.8cm} 40-60 stran

\vspace*{3mm}

\hspace*{-2.65cm} Klíčová slova: \hspace*{2.9cm} \parbox[t]{0.95\textwidth}{vizualizace dat, grammar of graphics, průzkumová analýza dat} \\

\vspace*{3mm}

\hspace*{-2.65cm}Doporučené zdroje informací: \\

\hspace*{-2.7cm} \parbox[t]{1.5\textwidth}{\small 1. WICKHAM, H. \textit{Ggplot2 : elegant graphics for data analysis}. Dordrecht: Springer, 2009. ISBN 978-0-387-98140-6.} \\

\hspace*{-2.75cm} Předběžný termín obhajoby: \hspace*{0.1cm} 2017/18 LS - FŽP \\

\vspace*{1mm}

\hspace*{-2cm} \parbox[t]{0.5\textwidth}{\centering Elektronicky zamítnuto: 25. 4. 2017 \\ \textbf{doc. Ing. Martin Hanel, Ph.D.} \\ Vedoucí katedry}
%------------------------------------------
\end{document}